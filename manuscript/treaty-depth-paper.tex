\documentclass[12pt]{article}

\usepackage{fullpage}
\usepackage{graphicx, rotating, booktabs} 
\usepackage{times} 
\usepackage{natbib} 
\usepackage{indentfirst} 
\usepackage{setspace}
\usepackage{grffile} 
\usepackage{hyperref}
\usepackage{adjustbox}
\usepackage{amsmath}
\usepackage{siunitx}
\setcitestyle{aysep{}}


\singlespace
\title{\textbf{The Sources of Alliance Treaty Depth}}
\author{Joshua Alley\footnote{Graduate Student,
Department of Political Science, Texas A\&M University.}}
\date{}

\bibliographystyle{apsr}

\begin{document}

\maketitle 

\doublespace 

\begin{abstract}
Why do states form deep alliances? 
Deep treaties add defense coordination and cooperation to promises of military support.
I argue that alliances between democracies tend to have greater treaty depth. 
Democratic states use treaty depth to reassure allies, because they are less likely to offer unconditional military support.
Using several statistical models, I show that decisions to include depth and unconditional military support in alliance treaties are correlated.
Moreover, I find that as the average democracy of alliance members at the time of formation increases, treaty depth increases, but the probability the treaty offers unconditional military support decreases. 
Thus, democracies do not uniformly offer limited commitments, but instead substitute between different sources of credibility. 
The argument and findings have implications for our understanding of how domestic politics shapes institutional design. 
\end{abstract}


\newpage 


\section{Introduction}


% Start with hook: maybe a story of a deep and shallow alliance 
Why do states make deep alliance treaties? 
While some alliance treaties include only a promise of military support, others supplement military support with commitments of extensive defense cooperation. 
To give a well-known example, the NATO treaty commits to setting up a formal international organization to govern the alliance. 


Treaty depth is a common policy choice, and it has important consequences. 
At least half of all ATOP alliances with offensive or defense obligations have at least one source of treaty depth.
Deep alliances encourage non-major power members to reduce military spending because treaty depth adds credibility.  
On the other hand, participation in shallow alliances increases military spending because states use military spending to hedge against abandonment.
The issue of military spending shows treaty depth affects alliance politics by shaping treaty credibility and the distribution of military spending burdens among members. 


% Describe question and contribution of the paper
Despite the consequences of alliance treaty depth, we know little about when states add depth to their alliances.\footnote{\citet{Mattes2012} examines a few possible causes of military institutionalization.}
In this paper, I explain when states make deep alliance treaties.
I argue that domestic politics shape how states use different sources of alliance credibility. 
Alliance negotiations center on whether prospective members will offer military support and conditions on that support \citep{Poast2019a}. 
When alliance members are unwilling to offer unconditional military support, depth provides an alternative source of reliability.%\footnote{Depth could also complement promises of unconditional military support.}


Because democracies tend to offer alliances with conditional obligations \citep{Mattes2012, Chibaetal2015}, they use treaty depth as a substitute source of reliability. 
Democracies make conditional promises because leaders fear audience costs of violation, and limited commitments are harder to violate.
At the same time, these limited alliances can increase allied fears of abandonment. 
Therefore, democracies use the costly provisions in deep alliances to reassure their partners.   


I test this argument with several statistical models and an illustrative case study of the North Atlantic Treaty Organization (NATO).
The statistical models connect decisions about treaty depth and unconditional military support and adjust for unobservable correlates of the two processes \citep{Braumoelleretal2018}. 
The case study checks the theoretical process and statistical results \citep{SeawrightGerring2008, Seawright2016}. 
I find consistent evidence that greater allied democracy at the time of alliance formation increases treaty depth and decreases the probability of unconditional military support. 


% Para on implications: sources of credibility and reassurance are connected
My argument and findings have important consequences for research on alliance treaty design. 
Existing scholarship considers parts of treaty design in isolation \citep{Benson2012, Mattes2012, Chibaetal2015}. 
My argument and findings show that sources of credibility in alliance treaty design are connected. 
Theories and models that address different alliance characteristics in isolation may generate misleading conclusions. 
For example, previous estimates of the connection between democracy and treaty depth produced null findings, in large part because previous models do not consider the association between depth and conditions on military support. 


% Paragraph on the importance of democracy in alliance pol
This paper adds to knowledge of how domestic politics affect alliance politics. 
Scholars have long acknowledged that democracy and alliances are connected \citep{LaiReiter2000, GiblerWolford2006, Warren2016, McManusYarhi-Milo2017}. 
Democratic commitments are often seen as more credible due to audience costs \citep{Gaubatz1996, DigiuseppePoast2016}, but this relationship is disputed \citep{GartzkeGleditsch2004}. 
Existing scholarship mostly suggests that democracies prefer limited commitments \citep{Mattes2012, Chibaetal2015}. 
My argument suggests that although democracies screen the scope of their commitments carefully, they form deeper alliances on other dimensions.  
The net effect of these differences on democratic credibility requires further research. 
My argument also contributes to a rich literature on domestic politics and international cooperation, e.g. \citep{DownesRocke1995, Fearon1998, Leeds1999, MattesRodriguez2014}. 


% roadmap for the paper 
The paper proceeds as follows. 
In the next section, I lay out the argument and hypothesis. 
Then I describe the data and research design. 
Following that, I provide empirical evidence from several statistical models and a case study of NATO. 
The final sections discuss the results and implications. 


\section{Argument}


In this argument, I start by defining treaty depth. 
Then, I show that treaty depth is understudied with a brief review of existing work on alliance treaty design. 
After that, I describe a general model of the process of alliance treaty negotiations. 
Finally, I describe how alliance negotiations between democracies often increase treaty depth. 


% define alliance treaty depth
Alliance depth is the extent of defense cooperation formalized in the treaty. 
Deep alliances require additional military policy coordination and cooperation. 
While shallow alliances stipulate more arms-length ties between members, deep treaties lead to closer cooperation through intermediate commitments that fall between treaty formation and military intervention. 
Defense cooperation in a deep alliance takes many forms. 
Allies can form an integrated military command, provide military aid, commit to a common defense policy, provide basing rights, set up an international organization or undertake companion military agreements. 


% note that I'm the first one to address this question: lit review. 
Depth is therefore an important part of alliance treaty design.
Alliances are elf-enforcing contracts or institutions \citep{Leedsetal2002, Morrow2000}.
Given external threats in an anarchic international system, states form alliances to aggregate military capability and secure their foreign policy interests \citep{Altfield1984, Smith1995, Snyder1997, FordhamPoast2014}. 


Potential alliance members can design a wide range of treaties \citep{Leedsetal2000, Leedsetal2002, Benson2012, BensonClinton2016}. 
Treaty design then shapes the costs and benefits of treaty participation. 
Beyond the benefit of potential military support, alliances also clarify international alignments \citep{Snyder1990} and support economic ties \citep{Gowa1995, Li2003, Long2003, Fordham2010, WolfordKim2017}. 
The costs of alliances include lost foreign policy autonomy \citep{Altfield1984, Morrow2000, Johnson2015}, and the potential consequences of opportunistic behavior. 
Opportunism in alliances includes abandonment, or the failure of alliance members to honor their commitments \citep{Leeds2003a, BerkemeierFuhrmann2018}, entrapment in unwanted conflicts \citep{Snyder1984}, and free-riding \citep{Morrow2000}.   


% Depth is understudied
Treaty design can address abandonment and entrapment, but the process of alliance treaty design is understudied \citep{Poast2019a}. 
\citet{Mattes2012} examined alliance treaty design by using symmetry of capability and history of violation to explain conditions on military support, issue linkages, and military institutionalization in bilateral alliances. 
She argues that all three design considerations increase treaty reliability.  
While checking the validity of their latent measure of treaty depth, \citet{BensonClinton2016} find that foreign policy agreement, major power involvement and treaty scope are all positively correlated with depth. 
Their latent measure of scope is based on the type of military support offered and conditions on that support. 


We have a better sense of when states offer conditional military support. 
\citet{Benson2012} shows that foreign policy disagreements and revisionist protege states increase the likelihood of limited military support.
\citep{Chibaetal2015} added to to existing work on limited obligations by showing that democracies are more likely to form alliances with conditional military support or consultation. 
Other work by \citet{Poast2012, Poast2013} establishes that states often use issue linkages to facilitate alliance formation. 


None of these works connect different sources of reliability.  
\citet{Mattes2012} studies military institutionalization and conditionality as separate sources of reliability. 
Her argument and research design treat depth and institutionalization as independent.
Though \citet{BensonClinton2016} use scope to predict depth and depth to predict scope, they do not consider the correlation between the two processes. 
Perhaps as a result of these theoretical and empirical choices, both works find no association between the democracy of alliance members and treaty depth. 
Because states can use different foreign policy instruments as substitutes or complements \citep{Starr2000, MorganPalmer2000}, treaty depth and conditions on military support are probably related. 
My argument builds on existing scholarship by placing depth and conditions on military support in a unified theoretical and empirical framework. 
I now describe the general process of the argument. 


\subsection{Alliance Negotiations and Obligations}


% process of alliance negotiations: once agreed on military support, need to bolster. 
Alliance treaty design is the result of negotiations between members \citep{Poast2019a}, where states determine conditions on military support and treaty depth.\footnote{The alliance formation stage is something I will need to model in a later robustness check.}
Both parts of treaty design address the benefits and costs of alliance participation. 
Depth and conditions on military support have different consequences, however. 


% Part 1: Establish military support and conditions on that support
Establishing if and when military support will be offered is the essential task for potential alliance partners. 
Promises of military intervention are the core of alliances. 
To form an alliance, the members must have sufficient overlap in foreign policy interests \citep{Morrow1991, Smith1995, FordhamPoast2014}, especially their proposed war plans \citep{Poast2019a}.  


Promises of military support in an alliance vary widely, however. 
The extent of shared foreign policy interests shapes whether alliance members offer unconditional or conditional military support.
Many alliances limit promises of intervention to particular regions, conflicts, or instances of non-provocation \citep{Leedsetal2000}. 
For example, if alliance members fear entrapment in unwanted conflicts, they will only offer military support in specific circumstances \citep{Kim2011, Benson2012}.\footnote{Such deliberate design of alliances means clear instances of entrapment are rare \citep{Kim2011, Beckley2015}.} 
Conditional treaties reflect less overlap in foreign policy interests. 


Offering unconditional military support is thus a strong signal of shared foreign policy interests. 
Attaching no conditions to a potential intervention means alliance members hazard the reputational \citep{Gibler2008, Crescenzietal2012} and audience \citep{Fearon1997} costs of treaty violation from many potential conflicts. 
Accepting these potential costs implies that conflict participation in many circumstances where the alliance could apply is acceptable.
This reflects less fear of entrapment and many shared foreign policy interests. 
Therefore unconditional alliances are a key source of credibility, because they are a costly signal of substantial foreign policy agreement. 


% Part 2: depth
Alliance partners also negotiate over how to reinforce promises of military support and put them into action. 
This determines the depth of the treaty and provides additional evidence of alliance reliability. 
Depth shapes the perceived reliability of an alliance by providing opportunities for states to fulfill treaty obligations in peacetime. 
Implementing deep treaty provisions can also enhance joint war-fighting. 


Because treaty depth and conditions on military support both add to alliance credibility, they are interdependent decisions.
Though depth and conditions on military support are related, they are not equivalent. 
Depth and military support conditions generate credibility through different costs. 


Conditions on military support do not change absent treaty renegotiation, and their costs are hypothetical unless the alliance is invoked.  
Alliance members and other states use the potential costs of military support to assess treaty reliability. 
Highly conditional alliances are less costly in expectation, which reduces their credibility. 


The costs of depth can be observed without invoking promises of military support. 
Therefore depth addresses time-inconsistency problems from changing foreign policy interests, which are a major threat to alliance fulfillment \citep{LeedsSavun2007}. 
Under a deep alliance treaty, members can use implementation of defense cooperation and policy coordination to assess allied reliability. 
Observing that alliance members adhere to peacetime promises indicates they will also honor promises of military support. 


This connection between depth and conditions on military support means that understanding how alliance member characteristics affect alliance treaty design must account for both sources of credibility.  
As I detail below, democracies are more likely to design alliances with conditional military support. 
As a result, democracies must then use greater treaty depth to address reliability concerns from these limited commitments.  
Alliances between democracies therefore include limited military support and high depth. 


\subsection{Alliances between Democracies}


% Start with a well-established result: democracy and conditional obligations
There is already ample evidence that democracies often prefer conditional military support. 
\citet{Mattes2012} and \citet{Chibaetal2015} both show that democracies are more likely to design conditional alliances. 
They attribute this finding to higher audience costs in democracies. 
Because democratic leaders face substantial audience costs from violating their international commitments, leaders design more limited commitments that are easier to fulfill. 
Based on this logic, even after accounting for the correlation between depth and scope, increasing the average democracy of alliance members at the time of formation should reduce the probability of unconditional military support.


\begin{quote}
\textsc{Unconditional Military Support Hypothesis: As the average democracy of alliance members at the time of formation increases, the probability that the alliance will include unconditional military support will decrease.}
\end{quote} 


% Move to treaty depth: look at it from a reliability perspective
Limiting alliance commitments through conditional military support reduces audience costs because it is easier for democratic leaders to backtrack on military support. 
Under a conditional alliance, states are not automatically obligated to intervene. 
As a result, it is possible for leaders to either claim that the conditions for intervention were not met, or that new information obviates the alliance commitment \citep{LevenduskyHorowitz2012}. 


Because allied states understand these limits on conditional alliances, these alliances will increase reliability concerns. 
Knowing that allies have given themselves an out will increase the fear of abandonment. 
This may undermine the effectiveness of deterrence from the alliance, as potential challengers pick on unreliable pacts \citep{Smith1995}. 
Therefore, democracies may need to find other ways to indicate their commitments are credible. 


Democracies can substitute treaty depth for unconditional military support. 
By including peacetime costs in a deep alliance, democratic states provide a different signal of reliability. 
Providing regular access to military aid, bases, or policy coordination indicates that alliance members are committed to the treaty. 
This increases allied confidence that democracies will honor their treaty obligations, even with conditional military support. 


Moreover, depth is a flexible way for democratic states to increase the perceived reliability of their alliances. 
Rather than make a fixed commitment of unconditional military support, democracies can shift their commitment to an alliance if electoral politics requires it.
For example, democratic leaders could reduce military aid to an ally if their constituents prefer a more limited foreign policy.  
Therefore, leaders are less constrained by the alliance choices of previous leaders, as they have some freedom to change how many peacetime costs they bear. 


Depth has more flexibility for democratic leaders because there are lower audience costs for not fulfilling peacetime promises. 
Audience costs, or disapproval of leaders not fulfilling their promises, increase as crises escalate \citep{Tomz2007}. 
While military intervention is costly and highly public, the day to day aspects of alliance management are less salient for the public. 
Any elite dissension about changes in commitment to a deep alliance is unlikely to translate into meaningful public opposition and electoral concerns. 


% express the key hypotheses
As a result of a preference for conditional military support and the lower audience costs of treaty depth, democracies will often design deep alliance treaties. 
As the average democracy of alliance members at the time of treaty formation increases, treaty depth should increase. 


\begin{quote}
\textsc{Treaty Depth Hypothesis: As the average democracy of alliance members at the time of formation increases, the depth of an alliance treaty will increase.}
\end{quote} 


% brief case illustration 
Many democratic alliances combine conditional military support and high treaty depth. 
Consider a 1960 defense pact between the United States and Japan (ATOPID 3375).
This alliance updated a 1951 defense treaty and included conditional obligations of military support. 
Promises of intervention are conditional on whether the fighting is taking place in East Asia. 
Moreover, the signatories promised action ``to meet the common danger'' if a member is attacked, which is not an explicit promise of military support. 
These kinds of limited promises are common in US alliances. 
The United States and Japan simultaneously formed a Security Consultative committee and permitted US troop bases in Japan, which are both sources of treaty depth. 


There is an important caveat to this argument--- I am interested in institutional design, not implementation.
Alliance treaty depth is not always implemented fully, as treaty aspirations are not fully realized, or work poorly. 
To give one example, several deep Arab alliances never realized their full intention due to internal political divisions.  
If states are to use treaty depth to increase treaty credibility, they must implement some of their alliance promises, however. 


Based on the argument and the case example, I expect that more democratic alliance membership will increase treaty depth. 
This occurs because democracies substitute depth for unconditional military support. 
In the next section, I describe how I test this claim about the association between democratic alliance membership and treaty depth. 




\section{Research Design}


My argument claims that treaty depth and conditions on military support are outcomes from related processes. 
I expect that democracy among alliance members will increase treaty depth and decrease the probability of unconditional military support. 
I examine these claims with a series of statistical models and a illustrative case study of NATO. 
I start by describing the key variables in the analysis. 
Then I provide more detail on the estimation strategy. 


% start with data
To examine my expectations that democracies tend to produce conditional alliances with substantial depth, I employ data on alliance treaty design from the Alliance Treaty Obligations and Provisions dataset \citep{Leedsetal2002}. 
My sample includes 289 alliances with either offensive or defensive obligations, which is the set of treaties with military support.\footnote{I plan to assess robustness to adjusting for non-random selection into alliances in the appendix.} 


I measured treaty depth using a semiparametric mixed factor analysis of eight ATOP variables \citep{Murrayetal2013}.\footnote{See \textbf{\href{https://github.com/joshuaalley/arms-allies/blob/master/manuscript/arms-allies-paper.pdf}{this paper}} for more details on the measure.}
This measure of depth is weighted combination of ATOP's defense policy coordination, military aid, integrated military command, formal organization, companion military agreement, specific contribution, and bases variables. 
Each of these individual indicators increases alliance treaty depth, but defense policy coordination and an integrated command have the largest positive association, as shown in the top panel of \autoref{fig:loadings-measure}. 


\begin{figure}[hbtp]
\centering
\includegraphics[width=0.95\textwidth]{../figures/loadings-measure.png}
\caption{Factor Loadings and posterior distributions of latent alliance treaty depth measure.}
\label{fig:loadings-measure}
\end{figure}


Based on these factor loadings, the measurement model predicts the likely value of treaty depth. 
The distribution of depth is summarized by the bottom panel of \autoref{fig:loadings-measure}. 
There is substantial variation in alliance treaty depth. 
Around half of all formal alliance treaties have an least some depth, and once states add some depth, there is a wide range of how much they include.


I measure alliance treaty depth in three ways.
First, I take the posterior mean of the latent depth posterior for each alliance. 
I also fit a statistical model that accounts for posterior uncertainty in the latent measure. 
Last, I measure alliance treaty depth with a dummy indicator of whether treaty depth in the alliance is above the median value. 
Results from these three measures are very similar. 


The other outcome variable is a dummy indicator of unconditional military support. 
Using ATOP's information on whether defensive or offensive promises are conditional on specific locations, adversaries, or non-provocation, I set this variable equal to one if the treaty placed no conditions on military support.
123 of 289 alliances in the data offer unconditional military support. 


The key independent variable is the average democracy of alliance members at the time of treaty formation. 
I use the POLITY measure of political institutions to measure democracy.\footnote{I also consider a ``strongest link'' measure, which is the maximum democracy value at the time of formation. See the appendix for results.} 
An alternative measure of allied democracy is a dummy variable which is equal to one if both alliance members have a polity score greater than 5. 
I prefer the average and maximum polity score measures because they translate better to multilateral alliances. 
Moreover, I expect that democracy in one member will affect the conditionality and depth of alliance treaties with less democratic partners.
Joint democracy is a slightly different concept. 



\subsection{Estimation Strategy}

I connect these measures with several statistical models. 
I start with separate models of treaty depth and unconditional military support. 
But because common unobserved factors may affect depth and conditionality, I specify a statistical model with two equations and correlated errors.
This approach is analogous to a bivariate probit model--- I do not use depth or unconditional military support as endogenous predictors of the other factor. 
One model predicts the probability of unconditional military support, and the other predicts treaty depth.


To model unconditional military support, I fit a binomial model with probit link function. 
The average democracy dummy is the key independent variable, and I also control for a range of other factors.
All of these variables are likely correlates of unconditional military support and allied democracy. 
Key controls include a dummies for asymmetric capability and symmetric alliances between non-major powers \citep{Mattes2012}\footnote{Major power alliances are the base category for these two variables.} and the average threat among alliance members at the time of treaty formation \citep{LeedsSavun2007}. 
I also control for foreign policy similarity \citep{Benson2012} using the minimum value of Cohen's $\kappa$ in the alliance \citep{Hage2011}.
Using the ATOP data \citep{Leedsetal2002}, I control for asymmetric treaty obligations, the number of alliance members, whether any alliance members were at war and the year of treaty formation. 
To capture the role of issue linkages in facilitating alliance agreements \citep{Poast2012, Poast2013}, I also include a dummy indicator of whether the alliance addressed economic issues.  
Last, I include a count of foreign policy concessions in the treaty, because concessions can facilitate alliance negotiations \citep{Johnson2015}. 


The model of treaty depth retains all of the above control variables and the average democracy variable. 
All these variables could conceivably alter the need for additional reliability from treaty depth. 
Modeling depth is more complicated because the latent measure is a skewed.
To facilitate model fitting, I transformed latent depth by transforming the variable to range between zero and one, then used a beta distribution for outcome.\footnote{I also considered log-logistic, Dagum and inverse Gaussian distributions for the outcome, but the beta model fit best, based on residuals and AIC.}
The flexibility of the beta distribution helps predict mean latent depth and facilitates fitting models that account for uncertainty in the latent measure, which I describe in more detail below. 


First, I fit these models separately. 
I fit the two models simultaneously using generalized joint regression modeling (GJRM) \citep{Braumoelleretal2018}.
GJRM uses copulas to model correlations in the error terms of multiple equation models, which makes it more flexible than parametric models and facilitates causal inference. 
Modeling any unobserved correlation between depth and unconditional military support facilitates accurate inferences about democracy and other covariates. 
Copulas are distributions over functions, and relax potentially problematic assumptions about the shape of the correlation in the error terms. 
I fit models with all copulas, and selected the best-fitting model using AIC, conditional on that estimator having converged.\footnote{GJRM is estimated with maximum likelihood, and diagnostics for the gradient as well as the information matrix suggest that the models converged.} 
The symmetric T copula provides the best model fit.
A third equation models possible heterogeneity in the error term using allied democracy and the start year of the alliance. 


GJRM also facilitates checking for non-linear relationships with smoothed terms. 
I estimated smoothed terms for democracy, mean threat and the start year of the alliance to capture potential non-linear relationships.\footnote{Fitting models with other continuous terms smoothed returned effective degrees of freedom = 1 for foreign policy similarity, which suggests a linear fit is equivalent. As the the results show, the smoothed term for democracy is highly linear.}  
Higher effective degrees of freedom on the smoothed term indicate non-linear relationships, and chi-squared statistics give approximate statistical significance. 
In the next section, I summarize the results of the analysis. 



\section{Results}


My findings are consistent with the claim that increasing democracy in an alliance leads to treaties with conditional support and greater depth. 
To begin, I offer some descriptive statistics.
Then, I present inferences from separate models. 
Last, I show evidence from two types of joint models. 


First, unconditional alliances tend to have lower average democracy values. 
The average of democracy among alliances with unconditional military support is -3.7. 
Conversely, average polity score in alliances with conditional obligations is -1.9.\footnote{Based on a t-test, the difference between these values is statistically significant.} 
There is also a slight positive correlation between average alliance democracy at the time of formation and treaty depth. 

% plot 
\autoref{fig:democ-combo} shows the mix of unconditional military support and high treaty depth in alliance design as democracy increases. 
In \autoref{fig:democ-combo}, democracy is on the x-axis, and treaty depth is on the y-axis.
Triangular points mark unconditional military support. 
At high levels of democracy, there are more conditional treaties, and most alliances have some depth. 


\begin{figure}[hbtp]
\centering
\includegraphics[width=0.95\textwidth]{../figures/democ-combo.png}
\caption{Presence of unconditional military support and depth in alliances from 1816 to 2016. This scatter plot shows mean latent treaty depth on the y-axis and the average polity score of founding alliance members on the x-axis. Triangular points mark treaties with unconditional military support, and the smoothed line trend line is fitted with linear regression.}
\label{fig:democ-combo}
\end{figure}


These descriptive results do not adjust for potential confounding factors, however.
Results from independent models of depth and conditionality support the hypotheses. 
\autoref{tab:separate-models} shows results from a beta model of treaty depth and a binomial model of unconditional military support with a probit link function. 
As expected, more democratic membership when the alliance forms is positively associated with treaty depth, but reduces the probability of unconditional military support. 


\begin{table}[!htbp] \centering 
\begin{tabular}{@{\extracolsep{5pt}}lcc} 
\\[-1.8ex]\hline 
\hline \\[-1.8ex] 
 & \multicolumn{2}{c}{\textit{Dependent variable:}} \\ 
\cline{2-3} 
\\[-1.8ex] & Latent Depth (rescaled) & Unconditional Military Support \\ 
\\[-1.8ex] & \textit{beta} & \textit{probit} \\ 
\\[-1.8ex] & (1) & (2)\\ 
\hline \\[-1.8ex] 
 Average Democracy & 0.026$^{}$ & $-$0.035$^{}$ \\ 
  & (0.001, 0.051) & ($-$0.072, 0.002) \\ 
  Foreign Policy Concessions & $-$0.054 & 0.004 \\ 
  & ($-$0.202, 0.093) & ($-$0.214, 0.222) \\ 
  Number of Members & 0.020 & $-$0.028 \\ 
  & ($-$0.005, 0.046) & ($-$0.074, 0.019) \\ 
  Wartime Alliance & $-$0.292$^{}$ & $-$0.950$^{}$ \\ 
  & ($-$0.630, 0.045) & ($-$1.568, $-$0.333) \\ 
  Asymmetric Obligations & 0.207 & 0.075 \\ 
  & ($-$0.122, 0.535) & ($-$0.427, 0.577) \\ 
  Asymmetric Capability & 0.361 & 0.607 \\ 
  & ($-$0.093, 0.816) & ($-$0.264, 1.479) \\ 
  Non-Major Power Only & 0.268 & 1.111$^{}$ \\ 
  & ($-$0.222, 0.758) & (0.224, 1.997) \\ 
  Average Threat & 0.995$^{}$ & 1.564$^{}$ \\ 
  & (0.161, 1.829) & (0.239, 2.888) \\ 
  Foreign Policy Disagreement & 0.185 & 0.368 \\ 
  & ($-$0.251, 0.622) & ($-$0.324, 1.060) \\ 
  Start Year & 0.005$^{}$ & 0.015$^{}$ \\ 
  & (0.001, 0.008) & (0.010, 0.021) \\ 
  Constant & $-$10.380$^{}$ & $-$31.678$^{}$ \\ 
  & ($-$16.984, $-$3.776) & ($-$43.109, $-$20.248) \\ 
 \hline \\[-1.8ex] 
Observations & 277 & 277 \\ 
Log Likelihood & 57.725 & $-$131.527 \\ 
\hline 
\hline \\[-1.8ex] 
\textit{Note:}  & \multicolumn{2}{r}{95\% Confidence Intervals in Parentheses.} \\ 
\end{tabular} 
 \caption{Results from separate models of treaty depth and unconditional military support in offensive and defensive ATOP alliances from 1816 to 2014. The depth model uses a beta regression to predict rescaled mean treaty depth. The model of unconditional military support is a binomial GLM with a probit link function.} 
  \label{tab:separate-models} 
\end{table} 


Although the results from separate models are as expected, these models do not account for unobserved correlations between depth and conditions on military support. 
This omission could easily affect inferences. 
I now report the results of the joint analysis of depth and unconditional military support. 
\autoref{fig:results-democ} plots the estimated effect of democracy on the two outcomes. 
The predicted effects of democracy on unconditional military support match the two hypotheses. 


\begin{figure}[hbtp]
\centering
\includegraphics[width=0.95\textwidth]{../figures/results-democ.png}
\caption{Predicted probabilities of unconditional military support and predicted changes in treaty depth across the range of alliance democracy. The line marks predicted values, and the shaded areas encapsulate the standard errors. Points mark observed values of average democracy. Predictions based on the smoothed terms from a joint generalized regression model.}
\label{fig:results-democ}
\end{figure}


The left-hand plot of \autoref{fig:results-democ} shows the association between the average democracy of alliance members and the predicted probability of unconditional military support. 
There is no clear association between average polity scores and the probability of conditional obligations. 
This finding contradicts existing results about democracy and conditional commitments \citep{Mattes2012, Chibaetal2015}.


The right-hand plot in \autoref{fig:results-democ} shows a positive relationship between democracy at the time of alliance formation and treaty depth. 
Substantial democracy in an alliance is associated with greater treaty depth. 
High autocratic alliances are usually more shallow, on the other hand. 
This trend of greater depth as the democracy of alliance members rises matches the treaty depth hypothesis. 


There are some other interesting inferences in the control variables, which I present in in \autoref{tab:gjrm-res}. 
This table contains results from both equations of the GJRM model, but omits the predictors of the error term correlation. 
I find that asymmetric capability in a treaty and symmetric non-major power alliances are more likely to have unconditional military support than major power alliances. 
I also find a negative association between the number of members in the alliance and the probability of unconditional military support. 

\begin{table}[ht]
\centering
\begin{tabular}{lrrrr}
  \hline
variable & Estimate.x & Std. Error.x & Estimate.y & Std. Error.y \\ 
  \hline
(Intercept) & -1.2892497 & 0.4751866 & -1.2725056 & 0.2467009 \\ 
  Economic Issue Linkage & 0.1848742 & 0.2136876 & 0.0725022 & 0.1513808 \\ 
  FP Concessions & -0.0912419 & 0.1230188 & -0.0408198 & 0.0817850 \\ 
  Number of Members & -0.0903415 & 0.0284946 & 0.0217991 & 0.0131066 \\ 
  Wartime Alliances & -0.5873107 & 0.3732149 & -0.0009069 & 0.1851105 \\ 
  Asymmetric Obligations & -0.0863500 & 0.2404688 & 0.1772261 & 0.1548069 \\ 
  Asymmetric Capability & 1.0102762 & 0.3865651 & 0.4655110 & 0.2136745 \\ 
  Non-Major Only & 1.7898074 & 0.4059615 & 0.2387748 & 0.2355886 \\ 
  FP Disagreement & 0.1550327 & 0.3585693 & 0.2742055 & 0.2137122 \\ 
  s(Avg. Democracy) & 1.0000001 & 0.0020490 & 1.1275745 & 7.0843557 \\ 
  s(Mean Threat) & 7.4405017 & 40.3457132 & 1.0000001 & 14.6307724 \\ 
  s(Start Year) & 5.4357206 & 42.7545506 & 4.7264598 & 41.1208709 \\ 
   \hline
\end{tabular}
\caption{Results from joint generalized regression model of treaty depth and unconditional military support. 
       All smoothed terms report the effective degrees of freedom and the chi-squared term. 
       The unconditional military support model is a binomial GLM with a probit link function. 
       The treaty depth model is a beta regression. 
       The error correlation between the two processes is modeled with a T copula.} 
\label{tab:gjrm-res}
\end{table}


There are also some notable patterns among the other variables in \autoref{tab:gjrm-res}. 
Asymmetric capability and the number of alliance members both increase depth. 
Last, threat and the year of alliance formation increase unconditional military support and treaty depth. 


There is no clear correlation in the errors of the data-generating processes for these two variables. 
Kendall's $\tau$ measures the correlation between the error terms, and it has a mean of -0.00526, while the 95\% confidence interval ranges from (-0.377,0.376).
This implies that unobserved confounding variables have little effect on either outcome. 
I find similar results with an alternative independent variable that measures the maximum polity score among all alliance members at the time of treaty formation. 

 
In a separate model, I consider how measurement uncertainty shapes inferences about the connection between non-major power alliances and treaty depth. 
The credible intervals in the bottom panel of \autoref{fig:loadings-measure} show, the latent measure of treaty depth has some uncertainty. 
This is a reasonable approximation of alliance politics, because alliance treaty depth is not observed with certainty. 
There are perceptible differences in treaty depth, especially once states add substantial depth to the treaty. 
Even so, the results from the analysis of mean treaty depth may overstate the effect of non-major power membership. 


To incorporate uncertainty over treaty depth, I fit a modification of the joint model. 
First, I created 1,000 datasets, one for each draw of the posterior distribution of the latent measure.
Then I fit the model of mean treaty depth to 500 randomly sampled datasets from those 1,000 to facilitate computation. 
For models of depth with uncertainty, I use BRMS \citep{Buerkner2017}. 
BRMS is an interface to STAN, a probabilistic programming language for Bayesian estimation \citep{Carpenteretal2016}. 
Joint Bayesian estimation has the flexibility to incorporate the probit and beta models and can be easily extended to account for uncertainty in the depth measure, but it does not allow correlated errors. 
Fitting the model sequentially to each dataset produces 500 separate models, which I combine into a single model by aggregating the posterior draws in a single posterior that accounts for uncertainty in the treaty depth measure.\footnote{Standard convergence diagnostics indicate convergence in all 500 models. Diagnostics like $\hat{r}$ are less useful for the full posterior, because some of the chains in the submodels do not overlap.}
This approach is analogous to common techniques for analyzing missing data, where multiple imputation generates uncertainty about the missing values \citep{Hollenbachetal2018imp}.
After multiple imputation, researchers fit a separate model to each imputed dataset and then combine the results. 


% Expand on these results later 
After accounting for uncertainty over treaty depth, I find a similar pattern. 
Allied democracy increases treaty depth, but decreases the probability of unconditional military support. 
These aggregate patterns give some sense of the theoretical process. 
To corroborate my theoretical claim that democracies substitute depth for unconditional support, I now offer a brief case study. 


\subsection{Case Study: NATO}


The case study examines the alliance negotiation process behind NATO, which is one of the most important alliances.
US foreign policy after World War II illustrates the tendency of democratic states to use treaty depth for reassurance, while also providing conditional obligations.  
Most US alliances have conditional promises of military support and some depth, and NATO fits this pattern. 


After the end of World War II, the US sought a way to protect Europe from the USSR. 
There were two challenges in negotiating the alliance, however.
First, as \citet{Poast2019a} details, NATO members disagreed over how to define the North Atlantic area, which was a key condition on military support. 
The US and other states fought over whether France's Algerian colony and Italy fell under the promises of support in the alliance. 


Second, the promises of military support in Article V of NATO are also rather contingent. 
Isolationists in the US Senate feared that an alliance would force America to intervene automatically if partners were attacked, bypassing the power of Congress to declare war \citep[pg. 280-1]{Acheson1969}.
Therefore Article V states that if one member is attacked the others ``will assist the Party or Parties so attacked by taking forthwith, individually and in concert with the other Parties, \emph{such action as it deems necessary} (emphasis mine).'' 
Military support was and is not guaranteed by the NATO treaty. 
Secretary of State Dean Acheson stated as much in a March 1949 press release, where he said that Article V ``does not mean that the United States would automatically be at war if one of the nations covered by the Pact is subject to armed attack'' \citep{Acheson1949}. 


The absence of automatic US involvement increased demand for reassurance by European allies. 
Europeans feared that if the Soviets invaded, the US would decide not to fight. 
Therefore, the US took other measures to reassure NATO allies. 
A 1951 presentation by Dean Acheson to Dwight Eisenhower argued European allies ``fear the inconstancy of United States purpose in Europe. ... These European fears and apprehensions can only be overcome if we move forward with determination and if we make the necessary full and active contribution in terms of both military forces and economic aid'' \citep[pg. 3]{Acheson1951}. 


The first part of reassurance was the creation of the Atlantic Council, which is an international organization and the only source of depth in the NATO treaty itself. 
The United States used participation in this organization and other efforts to coordinate collective defense to increase the perceived reliability of the alliance. 
By investing in the Atlantic Council and other sources of depth in NATO, the US addressed European fears of abandonment. 
For example, US officials thought that the British Foreign Minister viewed US provision of a supreme commander in Europe as ``a stimulus to European action'' \citep{Acheson1950}. 


Many Senators also opposed military aid to Europe \citep[pg 285]{Acheson1969}, which limited other efforts to add treaty depth. 
Bilateral agreements on troop deployments then became another instrument of reassurance. 
In 1950 the Germans formally requested clarification on whether an attack on US forces in Germany would be treated as an armed attack on the US- which the US said it would \citep[pg. 395]{Acheson1969}.  
Even after agreeing to deploy troops, US policymakers hoped Europeans would soon provide more for their own defense, while acknowledging the US ``should not dictate what they shall do'' \citep[pg. 2]{Johnson1950}. 
These bilateral arrangements and basing rights are not covered in the NATO treaty, but they added substantial depth.\footnote{This reveals an important limitation of my statistical approach.}  


% Sum up 
NATO negotiations reveal the tendency of democracies to add conditions on military support, but then turn to treaty depth to reassure their allies. 
The United States preferred conditional military support, but was willing to invest in deep military cooperation. 
The Atlantic Council and bureaucratic machinery of the NATO treaty are the basis for substantial defense cooperation. 
Therefore, though Article V is more limited than many realize, NATO is still a deep alliance. 


\section{Discussion}


% main evidence for an indirect effect
The findings from the statistical models and case study of NATO provide inconsistent evidence. 
I find regular evidence that democracies tend to form deep alliances.
Contrary to existing claims, I find mixed evidence that allied democracy decreases the probability of unconditional military support. 
This suggests a modification to the conventional wisdom that democracies make limited alliance commitments.
Democracies are no less likely to stipulate conditions on military support, but they are are more prone to increase the depth of their alliances. 


% Conclusion depends on how alliances and military spending are connected
The new conclusion about democracy and unconditional military support in this paper comes from a bivariate model of depth and unconditional military support. 
Inferences about democracy and other covariates depend on this theoretical and empirical choice to model depth and unconditional military support together. 
Again, treaty depth and unconditional military support are not included as independent variables for the other outcome. 
A fully recursive model that includes depth and unconditional military support as predictors of the other outcome requires valid instruments for identification. 
The independent models suggest that non-major power symmetry in an alliance could instrument for unconditional military support as a predictor of treaty depth. 
Given the potential for biased findings from a weak instrument or one that fails the exclusion restriction, and lack of clear instruments for treaty depth, I am hesitant to employ a fully recursive model.  



% Therefore, sources of alliance credibility are rather entangled
My findings show how different aspects of alliance treaty design are related. 
\citet{BensonClinton2016} use measurement models to show this in a descriptive fashion, but my findings give a sense of the process behind different combinations of depth and conditions on military support. 
Previous research on the causes of alliance treaty design \citep{Benson2012, Mattes2012, Chibaetal2015} focused on particular characteristics and treated them as independent. 
This approach can be informative, but it is incomplete. 


Although my argument and evidence offer some innovations, they have some limitations. 
First, I only examine variation in formal treaty design. 
This omits the implementation of alliance promises, which may be deeper or shallower than the treaty language alone implies. 
As the NATO case study shows, formal treaty depth is a guide for practical depth, but it may miss some differences between the alliance. 
At the same time, this study focuses on institutional design. 
Evolution and changes in practice are a useful subject for future inquiry. 


The claims in this paper have important consequences for several branches of scholarship. 
First, they speaks to debates about whether democracies make more credible commitments. 
This literature could benefit from incorporating these findings about alliance treaty design. 
The effect of democracy on credibility can be divided into conditions on military support, treaty depth, and the direct effect of institutions and domestic politics. 
These may have competing or conditional effects, and the net effect will require further explanation. 


In general, studies of the consequences of alliance participation should account for alliance design and membership. 
This study shows that alliance membership and design are closely related. 
Estimates of only member characteristics, or only treaty design, risk omitting correlated factors that also change the consequences of alliance participation. 


\section{Conclusion}



The key implication for scholarship is that alliance treaty design is the result of a series of correlated decisions. 
Rather than address each part of treaty design piecemeal, scholars should think about connections between the core aspects of treaty design. 
For example, future work might examine the connections between conditions on military support, treaty depth, and issue linkages.  
Debating what promises to include among these core obligations will be a crucial next step. 
Any discussion of alliance treaty design must also situate these decisions in the process of treaty negotiations.
As such, my findings support calls to correct the comparative neglect of negotiations in existing scholarship \citep{Poast2019a}. 


Last, alliances are an international institution, so some of the lessons from this work may apply to other institutions. 
There is an extensive literature on the design on international institutions \citep{DownesRocke1995, MartinSimmons1998, Koremenosetal2001, Koremenos2005, Thompson2010}.
Debates about the trade off between breadth and depth \citep{Downsetal1998, Gilligan2004} reflect correlated design decisions given potential institutional membership. 
Still, this literature could work to clarify how different aspects of institutional design are related. 


In conclusion, democratic alliance membership is a source of treaty depth. 
This may reflect a desire to substitute for . 
This shows how different sources of reliability shapes alliance treaty negotiations. 



\singlespace
 
\bibliography{../../../MasterBibliography} 





\end{document}