\documentclass[12pt]{article}

\usepackage{fullpage}
\usepackage{graphicx, rotating, booktabs} 
\usepackage{times} 
\usepackage{natbib} 
\usepackage{indentfirst} 
\usepackage{setspace}
\usepackage{grffile} 
\usepackage{hyperref}
\usepackage{adjustbox}
\usepackage{amsmath}
\usepackage{siunitx}
\usepackage{multirow}
\setcitestyle{aysep{}}


\singlespace
\title{\textbf{Democracy, Elections, and Alliance Treaty Depth}}
\author{Joshua Alley\footnote{Postdoctoral Research Associate,
University of Virginia.}}
\date{\today}

\bibliographystyle{apsr}

\begin{document}

\maketitle 

\doublespace 

\begin{abstract}
Why do states form deep alliance treaties, which reinforce military support promises with commitments of defense coordination and cooperation? 
I argue that democratic alliance leaders use treaty depth to make more credible alliance commitments in the face of leadership turnover. 
Competitive elections in democracies can empower leaders with less commitment to an alliance. 
Such leadership turnover threatens credible commitment by democracies. 
Therefore, incumbent leaders use treaty depth to make reduced alliance commitment more costly. 
Thus, competitive elections with the potential for leadership change in the most capable alliance member increases treaty depth.
I test this claim on offensive and defensive alliances from 1816 to 2007 and illustrate the theoretical mechanisms by examining NATO.
I find that electoral competition in the most capable alliance member increases treaty depth. 
The argument and findings provide insight into the connection between domestic politics and the design of international institutions. 
\end{abstract}


\newpage 


\section{Introduction}


% Start with question and motive 
Why do states make deep alliance treaties? 
Deep alliances formalize extensive defense cooperation by using additional military policy coordination and cooperation in to supplement promises of military intervention. 
While shallow alliances offer arms-length military support, deep treaties lead to closer ties between alliance members. 
Key sources of depth include an integrated military command, military aid, a common defense policy, basing rights, international organizations, and companion military agreements, and half of all offensive and defensive alliances\footnote{Treaties that promise active military support.} include at least one source of depth \citep{Leedsetal2002}. 
Despite the prevalence of alliance treaty depth, we have little idea when states add depth to their alliances. 


% Generalize: IO design is not just primary commitments, but how states support/implement their promises
Understanding the sources of deep alliances is worthwhile for two reasons.
First, depth shapes alliance credibility and the distribution of military spending among alliance members. 
Costly commitments in deep alliances increase the credibility of military support promises \citep{Morrow1994}, which then encourages non-major power members of deep alliances to reduce military spending \citep{Alley2020}.  
Second, the process behind alliance treaty depth may provide more general insights into international institution design. 
Members of international organizations often use costly commitments to support cooperation. 
For example, some trade agreements use third-party dispute settlement mechanisms to enforce agreements \citep{Smith2000}, or institutionalize formal monitoring arrangements \citep{Duretal2013}.  
Understanding when and why states employ close or arms-length cooperative commitments is therefore worthwhile. 


% Describe question and contribution of the paper
In this paper, I argue that democratic alliance leaders use treaty depth to increase the durability of their alliance commitments in the face of leadership turnover. 
Although the leader negotiating an alliance has high alliance commitment, competitive elections could empower less committed leaders and threaten credible commitments \citep{GartzkeGleditsch2004, LeedsSavun2007}.
In anticipation of this turnover threat, incumbent leaders can use treaty depth to make reducing an alliance commitment more difficult and costly. 
Therefore, I expect that electoral competition leads democracies to design deep alliance treaties. 


So long as voters support alliance formation, opposition leaders will have limited capacity to veto treaty depth. 
Executive constraints, another salient component of democratic institutions, do give opposition elites some influence to limit treaty depth at the margins, however.
Thus, the net effect of competitive elections and executive constraints in democracies is increased treaty depth in alliances with a democratic leader.  


% describe the test
I test the argument with a statistical analysis of offensive and defensive alliances from 1816 to 2007 and then examine the theoretical mechanisms in the context of NATO treaty design.
I find consistent evidence that greater electoral democracy in the alliance leader increases treaty depth. 
The results are robust to different model specifications and samples, as well as several multiple equation estimation techniques. 


% Two Paras on the gap I fill 
This paper contributes to knowledge of alliance treaty design and how domestic politics affects alliance choices.
Democracy affects alliance politics in many ways \citep{LaiReiter2000, GiblerWolford2006, Mattes2012a, Warren2016, McManusYarhi-Milo2017}. 
Connecting domestic electoral institutions and treaty depth adds to this scholarship and addresses an important gap.
The process of alliance treaty negotiation and design is understudied \citep{Poast2019a}, and there is little research on treaty depth because the nascent alliance treaty design literature emphasizes conditions on military support.
Existing research identifies entrapment concerns \citep{Kim2011, Benson2012} and democratic alliance membership \citep{Mattes2012, Chibaetal2015} as two sources of conditional obligations.\footnote{\citet{FjelstulReiter2019} supplement research on support conditions by arguing that democracies use incomplete alliance contracts to limit audience costs.} 


Two studies of alliance treaty design examine similar concepts to treaty depth, but both have important limitations.   
First, \citet{Mattes2012} finds that members of symmetric bilateral alliances where one partner has history of violation are more likely to use military institutionalization to increase treaty reliability. 
This paper makes an important contribution, but it only analyzes bilateral alliances and uses an ordinal military institutionalization measure by \citet{LeedsAnac2005} that understates variation in treaty depth.  
Second, while checking the validity of a latent measure of costly alliance obligations, \citet{BensonClinton2016} find that foreign policy agreement, major power involvement and treaty scope increase depth. 
Benson and Clinton define depth as how costly alliance obligations are in general, however, so their latent measure of depth includes secrecy and issue linkages and captures a broader concept than defense cooperation. 


Therefore, we still do not understand why alliance members employ treaty depth.
To address this lacuna, I consider how domestic political institutions and potential leadership turnover shape institutional design.
Democratic leaders add depth to alliances to maintain credible commitment by future leaders with different constituencies.  
Thus, leaders strategically employ different foreign policy tools within their institutional context \citep{HydeSaunders2020}. 


% implications: 
There are two primary implications of my argument and findings. 
First, they add important nuance to existing claims that democracies prefer limited alliance commitments \citep{Mattes2012, Chibaetal2015, FjelstulReiter2019}. 
Even as democracies screen the breadth of conditions on military support, they form deeper alliances in other ways.
Furthermore, the findings add to the rich literature on domestic politics and international cooperation e.g. \citep{DownesRocke1995, Fearon1998, Leeds1999, MattesRodriguez2014}. 
Just as democracies reassure partners with deep alliance commitments, electoral politics may push democracies to undertake strong international commitments with conditional primary obligations.\footnote{For example, in environmental agreements, democracies may prefer soft law commitments \citep{BoehmeltButkute2018}, but other aspects of these agreements may be deep and less salient.} 


% roadmap for the paper 
The paper proceeds as follows. 
In the next section, I lay out the argument and hypotheses. 
Then I summarize the data and research design. 
After this, I describe the results and detail U.S. alliance treaty design considerations in NATO negotiations.
In the final section I summarize the results and offer some concluding thoughts. 


\section{Argument}


In this argument, I first consider role of domestic politics in alliance negotiations. 
I then explain how leadership turnover in alliances threatens treaty credibility. 
Last, I consider how depth addresses credibility concerns from regular leadership turnover. 


% explain allliances
Alliances are self-enforcing contracts or institutions where states promise military intervention \citep{Leedsetal2002, Morrow2000}. 
When faced with external threats in an anarchic international system, states form alliances to aggregate military capability and secure their foreign policy interests \citep{Altfield1984, Smith1995, Snyder1997, FordhamPoast2014}.
Alliance participation has several costs and benefits.
Beyond the benefit of possible military support, alliances also clarify international alignments \citep{Snyder1990} and support economic ties \citep{Gowa1995, Li2003, Long2003, Fordham2010, WolfordKim2017}.  
The costs of alliance participation include opportunism and lost foreign policy autonomy \citep{Altfield1984, Morrow2000, Johnson2015}. 
Opportunism in alliances has three forms; abandonment of military intervention promises \citep{Leeds2003a, BerkemeierFuhrmann2018}, entrapment in unwanted conflicts \citep{Snyder1984}, and free-riding \citep{Morrow2000}.


% process of alliance negotiations: establishing a credible commitment. 
To form an alliance, states must have similar foreign policy interests \citep{Morrow1991, Smith1995, FordhamPoast2014}. 
\citep{Poast2019a} notes that agreement over a treaty in alliance negotiations depends on outside options and compatible war plans. 
Alliance treaties formalizing promises of military support take many forms \citep{Leedsetal2002, BensonClinton2016}. 
Treaty design shapes the costs and benefits of treaty participation and addresses potential opportunism.  
In the face of abandonment concerns, alliance members use formal commitments to increase the credibility of military intervention promises, as the costs of the alliance commitment provide indications of reliability \citep{Morrow2000}.


% Product of negotiations and domestic constraints
Most studies of alliance treaty design focus on negotiations between states, but as in other foreign policy domains, leaders in alliance negotiations are playing a two-level game \citep{Putnam1988}. 
Alliance agreements reflect both international and domestic political constraints on leaders. 
To give two examples, audience costs concerns shape conditions on military support in alliances between democracies \citep{Chibaetal2015, FjelstulReiter2019}, and whether democratic dyads form consultation or defense pacts depends on the risk of the current leader losing office \citep{Mattes2012a}. 


% credibility depends not just on the alliance, but domestic politics
Leaders must account for domestic politics in their efforts to establish credible treaty commitments.
When an alliance is invoked, leaders decide whether to honor or violate the treaty obligations. 
That decision depends in part on domestic concerns, as leaders consider and respond to the views of the coalition that put them in office. 


% leadership turnover- threat to reliability
Leadership turnover can therefore threaten alliance reliability.
New ruling coalitions often implement a different foreign policy \citep{Lobell2004, Narizny2007}.  
Changes in political regimes \citep{LeedsSavun2007}, or the coalitions backing a leader \citep{Leedsetal2009} both increase the risk of alliance treaty abrogation. 
Even as the incumbent negotiating an alliance supports treaty participation, if leadership change empowers an alliance skeptic, the state will be more likely to leave or violate an alliance.  


% institutions shape leadership turnover frequency and regularity
Domestic political institutions shape the frequency and regularity of leadership turnover by structuring how leaders are selected and replaced. 
Some institutions formalize frequent leadership changes, by regularly providing opportunities to select new leaders. 
Others institutions give make leadership without undoing the regime itself difficult at best. 


% democracies have higher risk of leadership change
Democracies have regular leadership turnover. 
Competitive elections for leadership are perhaps the defining feature of democracies, and elections often create different ruling coalitions. 
At a minimum, competitive elections with a viable opposition, multiple candidates and more than one legal party have potential for leadership change \citep{HydeMarinov2012}. 
This raises the prospect of cycling between different coalitions, which could have different material and ideological foreign policy interests. 
Such cycling may reduce the reliability of democratic alliance commitments through leadership turnover \citep{GartzkeGleditsch2004}.\footnote{See \citet{Gaubatz1996} for a skeptical take on the problem of cycling for democratic reliability.}


% democratic allies can understand this risk 
The allies of democracies should understand the risk of leadership turnover for treaty reliability. 
Although the leader they are negotiating with may support the alliances, future leaders may not. 
For states seeking a durable alliance commitment, new leaders in a democracy are a potential problem. 
If the alliance is to provide support over the long run, leadership turnover threatens alliance credibility.


% challenge for leaders
Thus, when democratic leaders negotiate an alliance treaty, they must consider what will happen when they leave office.
Establishing credible alliance commitments for democracies requires attention to leadership turnover. 
Given the stakes of an alliance, verbal assurances that future leaders will honor the treaty are likely inadequate.
Allies will require greater reassurance that leadership changes will not undo the alliance. 



\subsection{Treaty Depth, Leadership Turnover and Alliance Credibility}

% democracies can address 
Treaty depth is one way for democracies to address the threat of leadership turnover to treaty reliability. 
Depth generates credible commitments through sunk costs.
Regular sunk costs in deep alliances also address the time-inconsistency concerns at the heart of potential leadership changes.


% depth 
Sunk costs in treaty depth help alliance members establish credible commitments.
In a deep alliance, states supplement military support promises with commitments of peacetime cooperation like bases, military aid, policy coordination and formal institutions. 
For example, Treaty of Lisbon between European Union members reinforces defensive support promises with commitments to a common defense policy and funding a European Defense Agency. 
Depth adds to the perceived reliability of an alliance by providing opportunities for states to fulfill peacetime treaty obligations \citep{Morrow1994}. 
Implementing deep treaty obligations is a sunk cost signal of commitment.
Observing that alliance members adhere to peacetime promises suggests that they will also honor promises of military support.\footnote{\citet{LeedsAnac2005} find that alliances with high military institutionalization are less likely to be honored in war, which contradicts this claim that depth increases credibility. 
This finding has two limitations, however. 
First, observed challenges to an alliance may indicate lower credibility \citep{Smith1995}, so there is a selection problem with these estimates, which Leeds and Anac acknowledge. 
Second, their inferences depend on an ordinal measure of alliance treaty depth that limits variation in depth. 
When I employ a continuous latent depth measure in Leeds and Anac's model of alliance fulfillment in offense and defensive alliances, the depth coefficient is positive. 
I find similar results with another model of alliance fulfillment in war using data from \citet{BerkemeierFuhrmann2018}.
See the appendix for details.} 


% depth and turnover: 
Depth also addresses leadership turnover concerns in two ways. 
First, maintaining depth requires continued sunk costs.
Upholding aid, defense cooperation, and basing rights means allies must continue to invest in defense cooperation throughout the life of an alliance. 
When new leaders uphold deep alliance provisions, they send the same signals of reliability as their predecessors. 


% makes adjusting/reducing commitment more costly
Second, depth makes adjusting or reducing an alliance commitment more costly for new leaders. 
Formalizing defense cooperation means that leaders break treaty obligations if they reduce defense cooperation. 
This is in turn exposes new leaders to public and elite disapproval for violating international obligations, so long as the public is paying attention \citep{Slantchev2006, PotterBaum2014} and prefers compliance \citep{Chaudoin2014, KertzerBrutger2016}.


% address issue of rhetoric
These mechanisms are effective if a leader expresses rhetorical skepticism about an alliance. 
Relative to continuing costly cooperation in an alliance, talk is cheap. 
So long as a leader honors deep alliance obligations, their rhetoric will have a more limited impact on alliance credibility. 


% depth is part of domestic pol process 
Like every other part of an alliance, depth is subject to domestic political constraints. 
The ability of a leader to form a deep alliance depends on opposition and voter preferences.
So long as voters support alliance formation, they are unlikely to check treaty depth.\footnote{This implies that a selection process is present, which I account for in one statistical model in the empirical section.} 
Opposition elites can limit but not eliminate treaty depth when executive constraints give them the requisite influence. 


\subsection{Domestic Constraints on Depth} 


% alliance oppponents/out of power folks understand this
Alliance opponents and potential future leaders can anticipate the consequences of depth for their ability to draw back from alliance commitments. 
Whether these elites have leverage to check treaty depth during alliance negotiations depends on executive constraints. 
In addition to competitive elections, many democracies check the power of the executive with coequal branches of government, especially courts or legislatures. 
Executive constrains increase the potential influence of the opposition over alliance treaty design.  


% explain further- how
Executive constraints, such a legislative ratification of alliance treaties, give opposition members a say in treaty design. 


% voters are unlikely to punish depth, conditional on supporting the alliance. ]
Voters are less likely to oppose treaty depth. 
If voters support the alliance, they are unlikely to disapprove of treaty depth.
In fact, voters may see close cooperation with allies as appropriate. 
Therefore, voters are unlikely to restrain treaty depth, conditional on approval of the alliance itself. 


The limited role of voter concerns over the specifics of alliance treaty negotiations attenuates the ability of opposition elites to restrict depth and alliance commitments.
Opposition elites may be able to mobilize coalitions to oppose alliance formation, but if they are unable to block alliance formation, they will lack influence to do more than limit depth at the margins. 
Therefore, the positive effect of competitive elections is likely to outstrip the negative impact of executive constraints on treaty depth.
In the aggregate, democratic institutions will encourage deeper alliances as leaders seek to manage the problem of cycling and opposition leaders lack sufficient influence to eliminate treaty depth.  


% Sources of democratic influence in alliance negotiations. 
Of course, democracies may not get what they prefer in alliance negotiations. 
More capable states have greater influence on alliance negotiations \citep{Mattes2012}, because their partners lose out on foreign policy benefits without their participation.
The most capable state is often the alliance ``leader,'' and their preferences carry more weight. 
Therefore, I conceptualize democratic influence in terms of the political institutions of the most capable alliance member.\footnote{Emphasizing the influence of the largest state is further supported by evidence that the preferences of powerful capital-exporting states drive the design of bilateral investment treaties \citep{AlleePeinhardt2014}.} 



\subsection{Predictions}

% What is it exactly about democracies?
Electoral competition encourages incumbent leaders in democracies to form deep alliances.
Selecting leaders through elections raises the prospect of leadership turnover and cycling that limits the ability of a democracy to sustain credible commitments. 
Competitive elections are therefore the key characteristic of democracy for understanding treaty depth. 

% express the depth and elections hypothesis
Because depth helps address cycling concerns from leadership changes, democracies will often design deep alliance treaties. 
Therefore, competitive elections in the most capable alliance member at the time of treaty formation will increase treaty depth. 


\begin{quote}
\textsc{Competitive Elections Hypothesis: Competitive elections in the most capable alliance member at the time of formation will increase alliance treaty depth.}
\end{quote}   

% express the depth and constraints hypothesis 
On the other hand, executive constraints ---coequal branches of government with the executive--- should decrease treaty depth. 
Constraints give alliance opponents some influence to reduce treaty depth. 


\begin{quote}
\textsc{Executive Constraints Hypothesis: Executive constraints in the most capable alliance member at the time of formation will decrease alliance treaty depth.}
\end{quote}   


% deal with audience costs argument- quick compare/contrast 
This argument is slightly different from existing arguments linking democracy and alliance treaty design, which rely on audience costs. 
\citet{Mattes2012} and \citet{Chibaetal2015} attribute democratic states' preference for conditional military support to careful management of audience costs. 
In this argument, democratic leaders make limited commitments that are easier to fulfill in anticipation of removal from office from violating international commitments.\footnote{See \citet{HydeSaunders2020} for a more general framework of domestic constraints on foreign policy.} 
Limiting alliance commitments through conditional military support reduces audience costs because it is easier for democratic leaders to claim that the conditions for intervention were not met \citep{FjelstulReiter2019}, or that new information eliminates intervention obligations \citep{LevenduskyHorowitz2012}. 


% discuss these two arguments
The electoral competition argument and audience costs argument have some similarities and important differences. 
One reason treaty depth limits cycling in alliance commitments is that leaders may pay audience costs for violating deep alliance provisions. 
Rather than restrict audience costs to specific conditions, depth raises potential audience costs for future leaders with different interests. 
Depth allows leaders to make costly alliance commitments that precommit their successors to the alliance \citep{Mattes2012a}. 
In an audience costs argument, leaders fear that later treaty violations will weaken their future electoral support, and reason backwards from this to alliance treaty design. 
Electoral politics relies on incumbents anticipating a change in the ruling coalition, and seeking to maintain credible alliance commitments.\footnote{Leadership turnover might factor into democracies' tendency to offer conditional support. Leaders may use conditional support to ensure that their successors will fulfill alliance obligations.} 


% What about different types of autocracies?: can omit. 
My argument uses electoral competition to explain why democracies often form deep alliances. 
What about autocracies? 
There is substantial variation in how autocracies select leaders. 
For example, in single-party states, leaders rely on support from domestic elites, which affects their foreign policy decisions \citep{Weeks2014}.
Elite turnover in single party states usually keeps the same coalition in power, however.  
Leadership change in personalist regimes is more likely to be irregular and include regime or coalition changes.   
In general, no autocracy has the same concern of regular cycling between different coalitions as democracies.
Therefore, assuming that all autocracies are equivalent, relative to democracies is sufficient for testing the electoral politics argument.\footnote{Examining heterogeneity among autocracies in alliance treaty design is an interesting subject for future research, however.} 


% because depth and uncond milsup are related, correlated choices
I expect that democratic alliance leadership will increase treaty depth through competitive elections. 
In the next section, I describe my test of the association between democratic alliance leadership and alliance treaty design. 
I first describe the key variables in the analysis, then provide more detail on the estimation strategy.



\section{Research Design}



% start with data
To examine whether democracies are more likely to form alliances, I employ data from the Alliance Treaty Obligations and Provisions (ATOP) project \citep{Leedsetal2002}. 
The analysis focuses on the design of 289 alliances with either offensive or defensive obligations.\footnote{Results are robust to adjusting for non-random selection into alliances. See the appendix for details.}
First, I measure depth and conditions on military support in these alliances with active military support. 


I measure treaty depth with a semiparametric mixed factor analysis of eight ATOP variables \citep{Murrayetal2013}.
This measurement strategy has two advantages. 
First, unlike other measures, this approach captures the full spectrum of variation in defense cooperation across alliances.
This latent variable approach is more flexible than an ordinal measure \citep{LeedsAnac2005} and more focused on defense cooperation than another latent measure \citep{BensonClinton2016}.\footnote{See the appendix for results with measures by \citet{LeedsAnac2005} and \citet{BensonClinton2016}, which lead to similar inferences. I also discuss the relative advantages of my measure in more detail.}
Besides matching my conceptualization of treaty depth, the estimator relaxes distributional assumptions about the correlation between the factors and latent variable, making it more flexible and robust than other factor analytic models. 
Added flexibility from the semiparametric component aside, this model is a standard mixed Bayesian factor analysis. 
Based on the argument, I fit the model with a single latent factor, and the results corroborate this expectation.\footnote{This is a confirmatory factor analysis, not an exploratory analysis.}
An eigenvalue decomposition of the posterior mean correlations between the observed variables suggests that one latent factor explains most of the observed variation in the different sources of treaty depth.\footnote{Also, a model with one latent factor converges, while a model with two factors does not converge. Such model-fitting difficulties can indicate misspecification.} 


My depth measure is essentially a weighted combination of ATOP's defense policy coordination, military aid, peacetime integrated military command, formal organization, companion military agreement, specific contribution, wartime subordination and basing rights variables.
The weight of each variable is estimated by the measurement model, so it is driven by the data.  
All eight variables increase alliance treaty depth, but defense policy coordination and an integrated command add the most to depth, as shown in the top panel of \autoref{fig:loadings-measure}. 
Thus, policy coordination and formal organizational ties are the primary sources of treaty depth. 


\begin{figure}[hbtp]
\centering
\includegraphics[width=0.95\textwidth]{../figures/loadings-measure.png}
\caption{Factor loadings and posterior distributions of the latent alliance treaty depth measure. Estimates from a semiparametic mixed factor analysis of offensive and defensive ATOP alliances from 1816 to 2007.}
\label{fig:loadings-measure}
\end{figure}


% discuss/justify the loadings
These factor loadings are sensible. 
Defense policy coordination, peacetime integrated command structures and formal organizations all draw alliance members into closer peacetime defense cooperation. 
The other variables do not require as much direct cooperation, with the potential exception of bases.
Although bases are costly, they also serve multiple functions and may not require as much direct cooperation. 
In addition to deterrence and increasing alliance credibility, states often use basing obligations to project power, so bases are used for other purposes besides promoting cooperation between allies, while the other factors provide for more direct cooperation.  


The measurement model predicts the treaty depth of each alliance using the factor loadings. 
The bottom panel of \autoref{fig:loadings-measure} summarizes the posterior distributions of the latent treaty depth measure for every alliance in the data. 
There is substantial variation in alliance treaty depth. 
Around half of all alliance treaties have some depth, and depth varies widely across alliances.
In the analysis, I measure treaty depth using the mean of the latent depth posterior for each alliance. 
The posterior mean captures the central tendency of latent treaty depth, and I show in the appendix that results are robust to accounting for uncertainty in the latent measure. 


The other outcome variable is a dummy indicator of unconditional military support. 
Using ATOP's information on whether defensive or offensive promises are conditional on specific locations, adversaries, or non-provocation, I set this variable equal to one if the treaty placed no conditions on military support.
123 of 289 alliances offer unconditional military support. 


The key independent variable is an ordinal indicator of electoral democracy in the most capable alliance member during the year of alliance formation. 
I use the the Lexical Index of Electoral Democracy (LIED) \citep{Skaaningetal2015} to measure electoral democracy.
This ordinal measure assesses the extent of electoral democracy in a country based on six components and ranges from zero to six.  
States with minimally competitive multiparty elections for executive and legislative roles and universal suffrage have an index score of six.
Other states without elections, competition, or suffrage score lower on this scale.\footnote{In the appendix, I report findings with a dummy indicator of whether the most capable state has a LIED score of four or higher in place of the ordinal measure, which produces similar inferences.}
I rely on this measure for two reasons. 
First, unlike some measures of democracy, the LIED scale focuses on electoral democracy, so it measures the key concept of my argument.
Furthermore, the order of the scale places competitive elections before full suffrage, in contrast to other measures of electoral democracy, which give participation and electoral institutions equal weight. 
Widespread participation is only meaningful if leaders face some risk of removal through elections. 
I then code the alliance leader as the state with the largest composite index of national capabilities (CINC) score \citep{SingerCINC1988}, and measure their LIED score in the year the alliance formed.
The LIED of the most capable state therefore emphasizes the influence of the most capable alliance member and the prevalence of electoral democracy in that state.


Democracy has other components as well, and executive constraints are especially important in foreign policy \citep{MilnerTingley2015}. 
I adjust for executive constraints in the analysis, because constraints are positively associated with electoral democracy.
I set an executive constraints dummy equal to one if the executive constraints concept in the Polity data codes a state as having executive parity or subordination to other branches of government.
85 of the 289 alliances have such executive constraints on the leader of the most capable state.



\subsection{Estimation Strategy}



I use several bivariate statistical models to examine how democratic political institutions affect treaty depth.\footnote{Bivariate refers to a model means two outcome variables, not a model with one independent and one dependent variable.} 
Two factors encourage this research design choice. 
First, my argument expects that the data-generating processes behind treaty depth and unconditional support are related. 
Bivariate estimation also accounts for correlated errors because common unobserved factors may affect depth and conditionality \citep{Braumoelleretal2018}.
This modeling approach emulates the seemingly-unrelated regression model of \citet{FjelstulReiter2019}, who note that different alliance treaty design decisions are correlated. 
Independent univariate models assume that alliance treaty design decisions are uncorrelated, if this assumption is violated, biased estimates may result. 
My approach here is analogous to the well-known bivariate probit model, but it is not fully recursive, because I do not include depth or unconditional military support as endogenous predictors.\footnote{A fully recursive model requires two instruments for identification, and valid instruments are hard to find in alliance politics.}  
Instead, this model assumes that states make decisions about depth and conditions on military support at the same time. 


To predict unconditional military support, I use a binomial model with a probit link function. 
Modeling depth is more complicated because the latent measure is skewed.
To facilitate model fitting, I rescaled latent depth to range between zero and one and modeled it with a beta distribution.\footnote{I also considered log-logistic, Dagum and inverse Gaussian distributions for the outcome, but AIC and residuals showed that the beta distribution gave the best model fit.}
The flexibility of the beta distribution helps predict mean latent depth.\footnote{Using a beta distribution for the depth outcome also facilitates fitting models that account for uncertainty in the latent measure, which I include in the appendix. I also include univariate skew-t and skew-cauchy models of treaty depth without any transformation in the appendix, which find broadly similar results.} 
The alliance leader democracy measures are the key independent variables in both these model specificaitons. 


In the beta and probit models, I control for several correlates of treaty design and democratic institutions of the alliance leader. 
Key controls include dummy indicators of asymmetric alliances between non-major and major powers and symmetric alliances between major powers \citep{Mattes2012}\footnote{This leaves symmetric alliances between major powers as the base category for these two binary variables.} as well as the average threat among alliance members at the time of treaty formation \citep{LeedsSavun2007}. 
I also control for foreign policy similarity using the minimum value of Cohen's $\kappa$ in the alliance \citep{Hage2011}.
I draw on the ATOP data \citep{Leedsetal2002}, to adjust for for asymmetric treaty obligations, the number of alliance members and whether any alliance members were at war. 
To capture the role of issue linkages in facilitating alliance agreements \citep{Poast2012}, I include a dummy indicator of whether the alliance made any economic commitments.\footnote{In the appendix, I implement a trivariate model of treaty depth, unconditional military support and issue linkages, because issue linkages also increase treaty credibility \citep{ Poast2013}. This model still finds a positive relationship between electoral democracy and depth.}  
I adjust for a count of foreign policy concessions in the treaty, because concessions facilitate agreement in alliance negotiations \citep{Johnson2015}. 
Last, the model accounts for the role of time and the international context by using a non-linear smoothed term for the start year of the alliances to captures shift in the prevalence of deep or unconditional alliances over time.\footnote{I also check whether findings about democracy are driven the United States. See the appendix for results with an additional control for U.S. membership, which are similar to the inferences below.}


I use a generalized joint regression model (GJRM) \citep{Braumoelleretal2018} to combine the probit and beta specifications in a bivariate model.
This flexible estimator takes the probit and beta models and employs non-linear smoothed terms for threat and the start year of the alliance while estimating error term correlations. 
Adjusting for unobserved correlations between depth and unconditional military support ensures accurate inferences about democracy and other covariates.
GJRM uses copulas to model correlated errors in multiple equation models, which makes it more flexible than parametric models and facilitates causal inference. 
Copulas are distributions over functions, and they relax potentially problematic assumptions about the shape of the correlation in the error terms. 
Because controlling for error term correlations is crucial for inferences about correlated data-generating processes, considering a flexible set of functions to capture the error distributions ensures that the results are not an artifact of parametric assumptions. 
I fit models with every possible copula, and selected the best-fitting model using AIC, conditional on that estimator having converged.\footnote{GJRM uses maximum likelihood estimation, and diagnostics for the gradient as well as the information matrix suggest that the models behind all inferences in the paper and appendix converged.} 
The student-t copula maximizes model fit. 


A third equation in the GJRM estimator models heterogeneity in the error term correlations, which is important because depth and unconditional support could be positively correlated in some alliances, and negatively correlated in others. 
The argument also provides some insight for specifying this part of the model.  
If electoral democracy induces leaders to use depth in place of unconditional military support, electoral democracy should encourage negative error correlations. 
Conversely, if executive constraints encourage attempts to precommit successors with costly alliance commitments, constraints should make the error term correlation more positive.
Non-democracies might instead forgo both depth and unconditional support, or form alliances where depth complements unconditional obligations. 
Correlations in unobservable factors between treaty depth and unconditional promises of military spending could also vary with the international context.
For example, \citet{Kuo2019} shows how European politics encouraged the proliferation of secret alliances before World War I, and similar processes of emulation and diffusion may operate over time.
Last, multilateral alliances are likely to encourage deep and conditional obligations, as members hedge against entrapment and coordinate through treaty depth. 
Therefore, I use the start year of the alliance, the number of members and democratic institutions to predict.
To address the start year of the alliance, I include a smoothed term for the start year of the alliance in the error term equation.  


In summary, the GJRM model is a general and flexible way to simultaneously model different aspects of alliance treaty design.
Like the measurement model, it uses a semiparametric approach to relax potentially problematic assumptions about the distribution of underlying correlations. 
It also allows me to model the two outcomes with appropriate distributions. 
Before presenting inferences from this model, however, I discuss some descriptive statistics. 
I then use the bivariate GJRM model to estimate how electoral democracy and executive constraints affect alliance treaty design, followed by robustness checks with alternative measures of electoral democracy. 
The next section summarizes the results. 


\section{Results}


I find that electoral democracy in the most capable alliance member leads to deep and conditional alliance treaties, starting with descriptive statistics. 
Treaty depth and electoral democracy in the leading alliance member are positively correlated. 
Moreover, \autoref{fig:democ-combo} shows that average electoral democracy among the most capable member is greatest among deep and conditional alliances. 	
In \autoref{fig:democ-combo}, each quadrant corresponds to a combination of treaty depth and conditionality. 
To divide the continuous latent depth measure in two, I classified deep alliances as treaties with a latent depth score above the median value. 
On average, conditional and deep alliances have the highest electoral democracy in the most capable member. 
Conversely, unconditional alliances with little depth have low average electoral democracy.


\begin{figure}[hbtp]
\centering
\includegraphics[width=0.8\textwidth]{../figures/democ-combo.png}
\caption{Average electoral democracy score of the most capable member at the time of alliance formation across four groups of alliance from 1816 to 2007. Divisions between alliances based on unconditional military support and treaty depth. Darker quadrants mark higher average electoral democracy for that group of alliances, and the text in each box gives the precise value.}
\label{fig:democ-combo}
\end{figure}


These descriptive results do not adjust for potential confounding factors, however.
I now report the results of a bivariate model of depth and unconditional military support in \autoref{tab:gjrm-res}. 
This table presents results from both equations of the GJRM.\footnote{I mark smoothed terms with the letter s.} 


\begin{table}[ht]
\centering
\begin{tabular}{lrrrr}
  & \multicolumn{2}{c}{Uncond. Mil. Support} & \multicolumn{2}{c}{Latent Depth}\\ \hline
  & Estimate & Std. Error & Estimate & Std. Error \\ 
  \hline
  Executive Constraints & 0.7999527 & 0.2586244 & -0.3261589 & 0.1899146 \\ 
  Lexical Index of Democracy & -0.1731923 & 0.0487755 & 0.0959430 & 0.0386793 \\ 
  Economic Issue Linkage & 0.1285117 & 0.1690285 & -0.1126297 & 0.1341622 \\ 
  FP Concessions & -0.0825892 & 0.0795391 & 0.0128694 & 0.0661751 \\ 
  Number of Members & -0.1392637 & 0.0233697 & 0.0316344 & 0.0141233 \\ 
  Wartime Alliances & -0.6912033 & 0.1836950 & -0.0377874 & 0.1687670 \\ 
  Asymmetric Obligations & -0.1901903 & 0.1890448 & 0.2844494 & 0.1724515 \\ 
  Asymmetric Capability & 1.5219549 & 0.2982458 & 0.5252695 & 0.2170661 \\ 
  Non-Major Only & 2.2484581 & 0.3068160 & 0.1044143 & 0.2083990 \\ 
  FP Disagreement & 0.4266455 & 0.2313056 & 0.3240165 & 0.2014606 \\ 
  s(Mean Threat) & 7.7478269 & 124.7478875 & 1.0000048 & 24.9656885 \\ 
  s(Start Year) & 6.4781470 & 129.9064357 & 4.9620243 & 33.8646476 \\ 
  (Intercept) & -1.4686890 & 0.3025291 & -1.4048811 & 0.2724714 \\ 
   \hline
\end{tabular}
\caption{Results from joint generalized regression model of treaty depth and unconditional military support in 
         offensive and defensive alliances from 1816 to 2007. 
                     All smoothed terms report the effective degrees of freedom and the chi-squared term. 
                     The unconditional military support model is a binomial GLM with a probit link function. 
                     The treaty depth model is a beta regression. 
                     I model the error correlation between the two processes with a T copula.} 
\label{tab:gjrm-res}
\end{table}


These estimates are broadly consistent with the argument. 
As electoral democracy increases, alliance leaders are more likely to form deep alliances, and less likely to offer unconditional military support. 
Executive constraints has the opposite effect on alliance treaty design.
First, I find that executive constraints decrease treaty depth, which may reflect foreign policy constraints by other government actors with more foreign policy information than voters.
Second, domestic institutions with executive parity or subordination increase the probability of unconditional military support. 


Inferences about the control variables in this model are also interesting.
More alliance members and asymmetric capability both increase depth. 
Asymmetric alliances and symmetric non-major power alliances are more likely to include unconditional military support than symmetric major power alliances. 
I also find that alliances with more members or a member at war when the treaty formed have a lower probability of unconditional military support. 
Last, threat and the year of alliance formation increase unconditional military support and treaty depth, largely in a non-linear fashion.


To assess the substantive impact of elections, I estimated the difference between different levels of electoral democracy and no democracy using simulated coefficient vectors from the model. 
In the scenarios, I held all other variables at their mode or median and set executive constraints to one. 
I then varied the lexical index of electoral democracy across its full range and predicted treaty depth in seven hypothetical alliances. 
\autoref{fig:results-diff} plots the difference in predicted treaty depth or the probability of unconditional military support between six hypothetical alliances ranging from nominal to full electoral democracy and a baseline scenario with no democratic institutions. 


\begin{figure}[hbtp]
\centering
\includegraphics[width=0.95\textwidth]{../figures/results-diff.png}
\caption{Predicted difference in treaty depth and the probability of unconditional military support relative to a hypothetical alliance where the most capable state has no electoral democracy. Each scenario plots the estimated difference in treaty depth. All other variables held at their mean or median, expect for executive constraints, which is equal to one.}
\label{fig:results-diff}
\end{figure}


Greater electoral democracy has a clear substantive impact on alliance treaty depth and the probability of unconditional military support. 
Moving from no elections to a lexical index score of three, where a states has legislative and executive elections, adds between .01 and .08 to treaty depth, and decreases the probability of unconditional military support by between .09 and .28.
Full electoral democracy has a larger effect on both depth and unconditional military support. 
Alliances where the most capable state has full electoral democracy have between .04 and .21 more depth, all else equal. 
The decrease in the probability of unconditional military support is more uncertain, as it ranges from near -.18 to -.55. 
As rescaled depth and the probability of unconditional support both range between zero and one, these are meaningful substantive effects, though the magnitude of the electoral democracy effect is very uncertain. 


\autoref{fig:results-diff} shows that the extent of electoral democracy affects alliance treaty design.
When alliance leaders face electoral scrutiny, they are more inclined to form deep alliances with conditional military support. 
The above results depend in part on correlated errors between depth and unconditional support. 
In the GJRM model, the error term correlations are a function of democratic institutions and the international context. 
I now describe inferences about the error term correlations between treaty depth and unconditional military support. 


Contrary to my expectations, democratic institutions do not shape the magnitude and direction of the error term correlations between treaty depth and unconditional military support.
\autoref{tab:error-res} summarizes the estimates from the error term equation, which uses a smoothed term for the start year of the alliance and democratic institutions to predict the unobserved correlations between depth and unconditional military support. 
These estimates are on the scale of a parameter $\theta$ which captures the strength of the association of the errors. 
Both democratic institution coefficients are in the expected direction, but neither can be reliability distinguished from zero. 
Instead, the smoothed term for start years indicates that the relationship between depth and unconditional military support varies widely over time. 
The number of members also has a negative effect on the error correlations. 
The results in \autoref{tab:error-res} suggest that the error term correlation is not a fixed quantity. 
Rather, changes in the international context shape how treaty depth and unconditional military support are connected in alliance treaty design.


\begin{table}[ht]
\centering
\begin{tabular}{lrrrr}
  \hline
 & Estimate & Std. Error & z value & p-value \\ 
  \hline
  Executive Constraints & 0.8190942 & 0.6836838 & 1.1980601 & 0.2308936 \\ 
  Electoral Democracy & -0.0179937 & 0.1299139 & -0.1385048 & 0.8898415 \\ 
  Number of Members & -0.0280371 & 0.0300563 & -0.9328195 & 0.3509132 \\ 
  s(Start Year) & 7.9076932 & 8.4816153 & 62.2854436 & 0.0000000 \\ 
  Intercept & -0.3042886 & 1.3288257 & -0.2289906 & 0.8188762 \\ 
   \hline
\end{tabular}
\caption{Error term correlation equation estimates from a joint generalized regression model of treaty depth and unconditional military support. 
                    Estimates are on the scale of $\theta$, which is then converted into a Kendall's $\tau$ correlation coefficient. 
                    } 
\label{tab:error-res}
\end{table}


I find that increasing the lexical index of electoral democracy increases depth but decreases unconditional military support. 
There are other ways to operationalize electoral democracy, however. 
I now show that my results are largely robust to changing the measure of electoral democracy. 


\subsection{Alternative Measures of Electoral Competition}


Other democracy measures emphasize the presence of open and competitive elections. 
As such, these measures should generate similar inferences about the connection between democratic institutions and alliance treaty design. 
In this section, I assess three such measures, including one that conceptualizes democratic influence in terms of the proportion of alliance members with high electoral democracy. 


The first measure of electoral competition is an dummy measure based on the Polity data's executive recruitment concept.  
When the most capable alliance member has competitive elections, I set this dummy variable to one, and zero otherwise. 
72 alliances have a leader with electoral competition, according to the Polity criteria. 


The other measure of electoral democracy builds off of the concept of polyarchy \citep{Dahl1971}, which includes both contestation and electoral inclusiveness. 
Polyarchy relies on both the opportunity and freedom of actors to compete in elections for leadership, as leaders must address voters' preferences to avoid being removed from office. 
It places more weight on inclusive participation in politics beyond elections than the lexical index of democracy, however. 
To measure polyarchy, I use the polyarchy measure of electoral democracy from the Varieties of Democracy dataset \citep{Teorelletal2016}.
As with the lexical index of democracy, I measure polyarchy in the most capable alliance member in the year the treaty formed.

 
I then fit the same bivariate models of treaty depth and unconditional military support, and replaced the lexical index of electoral democracy with the polity and polyarchy measures.
I also changed the copulas if needed to maximize model fit and added a dummy indicator of open political competition from the Polity data to the polity model, because such competition is positively correlated with electoral recruitment of leaders. 
All of these models retain the executive constraints variable from Polity, as this variable is correlated with electoral 
competition and alliance treaty design. 


As a further check, I translated my electoral democracy measure into an alternative conceptualization of democratic influence--- the proportion of democracies in the alliance.
\citet{Chibaetal2015} use the proportion of democracies as their key independent variable, and code this variable as the share of alliance members with a Polity score above 5 when the alliance formed. 
A greater proportion of electoral democracies should also increase alliance treaty depth, as democratic concerns have more weight, and the leading state is more likely to be democratic. 
I express the proportion of electoral democracies as the share of alliance members with a score of four or higher on the electoral index of democracy. 
I also control for the share of alliance members with executive constraints. 
Because democracies cooperate more with one another \citep{Leeds1999}, the proportion measures are positively correlated with the democratic institutions of the most capable alliance member. 


In \autoref{fig:results-other-democ}, I plot the substantive impact of moving from the minimum to the maximum of the two electoral competition measures and the proportion of alliance members with high electoral democracy. 
As in \autoref{fig:results-diff}, these substantive effect calculations hold all other variables in the model constant, and I fix executive constraints to one. 
Each figure in the plot contains three estimates--- the predicted outcome with the key independent variable at its minimum value, the same predictions with the variable at its maximum, and the difference between the two scenarios. 


\begin{figure}[hbtp]
\centering
\includegraphics[width=0.95\textwidth]{../figures/results-other-democ.png}
\caption{Predicted treaty depth and probability of unconditional military support in offensive and defensive alliances from 1816 to 2007, all else equal besides an indicator of electoral competition. For each measure of electoral democracy, this figure shows the estimated treaty depth or probability of unconditional military support when electoral democracy is at maximum or minimum, along with the difference between the two scenarios.}
\label{fig:results-other-democ}
\end{figure}


Results with proportion of alliance members that have high electoral democracy partially match my hypotheses.  
The predicted difference in treaty depth between the low and high extent of electoral competition among alliance members ranges between .03 and .34.
More alliance members with high electoral democracy is also consistent with a reduced probability of unconditional military support. 


Inferences about the association between the polity measure of electoral competition and alliance treaty design are also consistent with the argument that elections lead to greater alliance treaty depth.
Moving from no electoral competition to competition increases depth by between .03 and .24, which is a large effect relative to the range of rescaled treaty depth. 
The same shift in electoral competition reduces the probability of unconditional military support by between .2 and .74, so the magnitude of that effect is more uncertain.  


The Varieties of Democracy polyarchy measure gives slightly different inferences. 
As expected, greater electoral democracy increases treaty depth by between .17 and .43. 
There is no clear difference in the probability of unconditional military support, however.
The polyarchy coefficient behind these substantive estimates is positive, but not clearly so.  
This finding could be explained by the way the Varieties of Democracy project weights freedom of information and association, as well as electoral competition and suffrage. 
The polyarchy measure is a complicated weighted average that penalizes weaknesses in freedom of information and association for electoral democracy more than other measures.
Thus, polyarchy places more weight on political participation in general than the other measures of electoral democracy. 
Aggregation choices in composite indicators of democracy may therefore affect overall inferences about the relationship between democracy and institutional design.\footnote{See the appendix for results with the Polity measure of democracy, which are broadly consistent with my findings here.}



The results of these statistical models imply that democratic institutions impact alliance treaty design. 
Electoral competition pushes alliances members to employ treaty depth, but reduces the probability of unconditional military support.  
To illustrate the theoretical process more directly, I now examine the North Atlantic Treaty Organization (NATO).


\subsection{NATO Treaty Design}


I use NATO to show the theoretical mechanisms for two reasons. 
First, the process behind NATO applies to multiple alliances, as other US alliance treaties have similar designs. 
Second, NATO is the most important alliance in international politics, as it has a crucial role in the structure of international relations by tying the United States to Europe. 
Because NATO is an exceptionally durable and consequential alliance, understanding how the treaty formed is worthwhile. 


After World War II, the United States sought a way to protect Europe from the USSR. 
Despite acute security concerns, criticism from opposition politicians led the United States to offer conditional military support. 
As \citet{Poast2019a} details, NATO members disagreed over how to define the North Atlantic area, especially with reference to France's Algerian colony and Italy, as the North Atlantic area was a key condition on military support. 
Furthermore, active military support from NATO members depends on domestic political processes.\footnote{\citet{Benson2012} calls this commitment a ``probablistic'' obligation.} 
Isolationists in the US Senate feared that an alliance would force automatic intervention, bypassing the power of Congress to declare war and engaging the US in unwanted conflicts \citep[pg. 280-1]{Acheson1969}.
Therefore, Article V of the NATO treaty states that if one member is attacked the others ``will assist the Party or Parties so attacked by taking forthwith, individually and in concert with the other Parties, \emph{such action as it deems necessary} (emphasis mine).'' 
Military support was and is not guaranteed by Article V, and US policymakers used this limited commitment to sell NATO to the public. 
In a March 1949 press release to the public, Secretary of State Dean Acheson said that Article V ``does not mean that the United States would automatically be at war if one of the nations covered by the Pact is subject to armed attack'' \citep{Acheson1949}.
This claim and the emphases of the press release show that promises of military support were salient to the US public, which was a key rationale for limited promises of military support. 


Military support from Article V did not assuage European fears that if the Soviets invaded, the United States would abandon them.
To increase the credibility of NATO, the United States took other measures.  
A 1951 presentation by Dean Acheson to Dwight Eisenhower argued that European allies ``fear the inconstancy of United States purpose in Europe. ... These European fears and apprehensions can only be overcome if we move forward with determination and if we make the necessary full and active contribution in terms of both military forces and economic aid'' \citep[pg. 3]{Acheson1951}. 
To start, the US supported the Atlantic Council, an international organization and the main source of depth in the NATO treaty. 
The United States used the Atlantic Council to coordinate collective defense and increase the perceived reliability of the alliance. 
By investing in the Atlantic Council and related joint military planning, the United States addressed European fears of abandonment. 
For example, US officials thought that the British Foreign Minister viewed US provision of a supreme commander in Europe as ``a stimulus to European action'' in NATO \citep{Acheson1950}. 


Policymakers then used defense cooperation with allies to justify NATO participation by arguing that it would facilitate more efficient defense spending. 
In an interview with NBC on March 29, Ambassador at Large Philip Jessup argued that ``One defense program is cheaper and more effective than a dozen national programs. It entails the pooling of information, a joint defense strategy and a pooling of military resources for defense.''
This claim was meant to assuage concerns that NATO would reduce the US ``peace dividend'' after World War II. 


NATO also illustrates that executive constraints might limit treaty depth. 
Many Senators also opposed military aid to Europe \citep[pg 285]{Acheson1969}. 
Thus, legislative constraints on the executive branch reduced the formal depth of NATO relative to what many ambassadors preferred \citep[pg 277]{Acheson1969}, which matches the statistical inference about executive constraints and treaty depth.  
Bilateral agreements on troop deployments thus became another instrument of reassurance. 
In 1950 the Germans formally requested clarification on whether an attack on US forces in Germany would be treated as an armed attack on the United States- and US policymakers said that it would \citep[pg. 395]{Acheson1969}.  
These bilateral arrangements and basing rights are not covered in the NATO treaty, but they added substantial depth.\footnote{This reveals a potential limitation of the statistical models.}  


% Sum up 
In NATO, electoral concerns and anticipation of opposition criticism led the United States to offer conditional military support, but did not inhibit deep military cooperation, which helped reassure European allies. 
Limits on the promises of military support were a salient part of public discussions in the NATO treaty, while the Atlantic Council had a smaller role in public discourse, and policymakers attempted to sell such cooperation as source of efficient defense spending. 
The Atlantic Council and associated bureaucracy are the formal core of substantial defense cooperation. 
NATO negotiations show how a democratic alliance leader used treaty depth to reassure their allies, rather than unconditional military support. 
In the next section, I summarize some implications of the results and offer concluding thoughts. 



\section{Discussion and Conclusion} 


% main evidence summary
In summary, the findings from the statistical models generate fairly consistent evidence for the hypotheses, and the NATO illustration suggests that the theoretical mechanisms are plausible. 
Across multiple models and measures, electoral democracy is positively correlated with treaty depth.  
Because depth is a less transparent source of alliance credibility, democratic leaders use depth to increase the credibility of alliance commitments, while avoiding more transparent credibility sources like unconditional military support.


% limitations
My argument and evidence have two limitations.
First, I only examine variation in formal treaty design. 
This omits treaty implementation, which can diverge from the formal commitment.   
Formal treaty depth often reflects practical depth, but it may understate some differences between alliances. 
Changes in realized alliance depth are a useful subject for future inquiry, but will require extensive data collection.
Second, I examine 280 alliances, so the sample size is limited. 
Inferences from small samples can be more sensitive to model and data changes. 


Shortcomings aside, this paper has four implications for scholarship. 
First, alliance treaty design is often driven by domestic political considerations. 
Attempts to remain in office and avoid opposition criticism in electoral politics encourage democratic leaders to design deep alliance treaties with conditional promises on military support. 


Second, different aspects of alliance treaty design are related \citep{FjelstulReiter2019}. 
As states attempt to make credible alliance commitments, they can employ a range of treaty obligations. 
Furthermore, the connection between depth and unconditional support varies with the international context. 
Thus, studying individual aspects of alliance treaty design in isolation leads to incomplete portrayals of the treaty design process. 


Third, democracies do not make fully limited alliance commitments with conditional obligations and no depth.
Even if democracies impose conditions on military support, treaty depth adds costly obligations.
As a result, democracies make robust alliance commitments in one way, and limited commitments in another. 


Last, some of the lessons from this work might apply to the design on international institutions in general \citep{DownesRocke1995, MartinSimmons1998, Koremenosetal2001, Thompson2010}.
In the same way that democracies use depth to support allies while managing electoral politics, democracies may undertake deep international commitments in ways that limit electoral scrutiny. 
The same mix of limited core obligations and deep cooperation may characterize other international institutions with democratic leadership. 


The findings raise at least two questions for future research.  
For one, they address debates about whether democracies make more credible commitments. 
Even if conditional military support reduces the credibility of democratic alliances, treaty depth has the opposite effect. 
The net effect of democracy on alliance credibility therefore includes conditions on military support, treaty depth, and the direct effect of democratic institutions and domestic politics. 
These three mechanisms may have competing or conditional effects, which could explain mixed findings about the credibility of democratic commitments \citep{Schultz1999, Leeds1999, Thyne2012, DownesSechser2012, PotterBaum2014}.
Future research should combine the components of democracy and democratic alliances to asses the net effect of democracy on credible commitment in international relations. 


Scholars should also consider how alliance treaty design varies across different types of autocracies. 
The extent and sources of political competition in autocracies varies widely. 
Differences in who selects leaders and what information those actors have about foreign policy \citep{Weeks2008} may help explain alliance treaty design.
For example, personalist leaders with few public or elite constraints on their foreign policy may design alliances with depth and unconditional military support. 
Single party states where leaders face an informed elite may prefer fully limited commitments with shallow and conditional obligations. 


In conclusion, electoral democracy encourages democracies to use treaty depth to increase the credibility of their alliances. 
Deep alliances reassure democracies' alliance partners while limiting electoral criticism of alliances. 
By shaping leaders' foreign policy audiences, domestic political institutions influence how states build credibility into alliance treaties.




 
\bibliography{../../../MasterBibliography} 





\end{document}
